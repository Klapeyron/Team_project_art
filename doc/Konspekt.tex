\documentclass[11pt]{article}
 
\usepackage[T1]{fontenc}
\usepackage[polish]{babel}

\usepackage{fontspec}

\begin{document}

\title{}
\begin{center}
\Large{PLAN PROJEKTU ZESPOŁOWEGO}
\end{center} 

\textbf{Grupa:} Bartłomiej Ankowski, Tomasz Bartos, Wiktor Gerstenstein

\textbf{Temat projektu:} Rummy gin expert bot system

\section{Cele projektu}
Celem projektu jest stworzenie programu komputerowego grającego w grę karcianą ''Remik'' na portalu internetowym \textit{kurnik.pl}.

\section{Aspekty badawcze projektu}
Projekt ma na celu sprawdzenie, czy program komputerowy na podstawie niepełnej informacji o stanie gry  jest w stanie, bazując na statystyce i algorytmach sztucznej inteligencji osiągać takie same bądź lepsze wyniki co gracz.

\section{Praktyczne wykorzystanie}
Program będzie umożliwiał graczowi osiąganie lepszych wyników w grze oraz samoczynne zdobywanie punktów rankingowych.

\section{Początkowa literatura}
\begin{itemize}
\item Gary Bradski and Adrian Kaehler: „Learning OpenCV”, O’Reilly Media, Sebastopol 2008
\item Ryszard Tadeusiewicz: „Odkrywanie właściwości sieci neuronowych”, Polska Akademia Umiejętności , Kraków 2007
\item Ray Rischpater, „Application Development with Qt Creator - Second Edition”, Packt Publishing, 2014
\end{itemize}

\section{Harmonogram i realizacja projektu}

\subsection{Podział obowiązków}
\begin{itemize}
\item Modułu sterowania grą - \underline{Bartłomiej Ankowski}
\item Modułu decyzyjny - \underline{Wiktor Gerstenstein}
\item Modułu odpowiedzialny za rozpoznawanie stanu stołu - \underline{Tomasz Bartos}
\end{itemize}

\subsection{Roadmap projektu}
\href{https://trello.com/b/PmTsc0OK/team-project-art-roadmap}{Roadmap projektu znajduje się w ponizszym linku}

\end{document}
